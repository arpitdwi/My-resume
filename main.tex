%-------------------------
% Resume in Latex
% Author : Arpit Dwivedi
%
% 
%------------------------

\documentclass[letterpaper,11pt]{article}

\usepackage{latexsym}
\usepackage[empty]{fullpage}
\usepackage{titlesec}
\usepackage{marvosym}
\usepackage[usenames,dvipsnames]{color}
\usepackage{verbatim}
\usepackage{enumitem}
\usepackage[hidelinks]{hyperref}
\usepackage{fancyhdr}
\usepackage[english]{babel}
\usepackage{tabularx}
\usepackage{xcolor}
\usepackage{fontawesome5}


\input{glyphtounicode}

% -------------------- FONT OPTIONS --------------------
% sans-serif
% \usepackage[sfdefault]{roboto}
% \usepackage[sfdefault]{noto-sans}
% serif
% \usepackage{charter}

\pagestyle{fancy}
\fancyhf{} % clear all header and footer fields
\fancyfoot{}
\renewcommand{\headrulewidth}{0pt}
\renewcommand{\footrulewidth}{0pt}

% Adjust margins
\addtolength{\oddsidemargin}{-0.5in}
\addtolength{\evensidemargin}{-0.5in}
\addtolength{\textwidth}{1in}
\addtolength{\topmargin}{-0.5in} 
\addtolength{\textheight}{1.0in}

\urlstyle{same}

\raggedbottom
\raggedright
\setlength{\tabcolsep}{0in}

% Section formatting
\titleformat{\section}{
  \vspace{-5pt}\scshape\raggedright\large
}{}{0em}{}[\color{black}\titlerule \vspace{-5pt}]

% Subsection formatting
\titleformat{\subsection}{
  \vspace{-4pt}\scshape\raggedright\large
}{\hspace{-.15in}}{0em}{}[\color{black}\vspace{-8pt}]

% Ensure that generate pdf is machine readable/ATS parsable
\pdfgentounicode=1

% -------------------- CUSTOM COMMANDS --------------------
\newcommand{\resumeItem}[1]{
  \item\small{
    {#1 \vspace{-2pt}}
  }
}

\newcommand{\resumeSubheading}[4]{
  \vspace{-2pt}\item
    \begin{tabular*}{0.97\textwidth}[t]{l@{\extracolsep{\fill}}r}
      \textbf{#1} & #2 \\
      \textit{\small#3} & \textit{\small #4} \\
    \end{tabular*}\vspace{-7pt}
}

\newcommand{\resumeSubSubheading}[2]{
    \item
    \begin{tabular*}{0.97\textwidth}{l@{\extracolsep{\fill}}r}
      \textit{\small#1} & \textit{\small #2} \\
    \end{tabular*}\vspace{-7pt}
}

\newcommand{\resumeProjectHeading}[2]{
    \item
    \begin{tabular*}{0.97\textwidth}{l@{\extracolsep{\fill}}r}
      \small#1 & #2 \\
    \end{tabular*}\vspace{-7pt}
}

\newcommand{\resumeSubItem}[1]{\resumeItem{#1}\vspace{-4pt}}
\newcommand{\resumeSubHeadingListStart}{\begin{itemize}[leftmargin=0.15in, label={}]}
\newcommand{\resumeSubHeadingListEnd}{\end{itemize}}
\newcommand{\resumeItemListStart}{\begin{itemize}}
\newcommand{\resumeItemListEnd}{\end{itemize}\vspace{-5pt}}

\renewcommand\labelitemii{$\vcenter{\hbox{\tiny$\bullet$}}$}

\setlength{\footskip}{4.08003pt}

% -------------------- START OF DOCUMENT --------------------
\begin{document}

% -------------------- HEADING--------------------
%\begin{flushright}
%  % \vspace{-4pt}
%  \color{gray}
%  \item
%  Last Updated on January 12th, 2024
%\end{flushright}

\vspace{-5pt}

\begin{center}
    \textbf{\Huge \scshape Arpit Dwivedi} \\ \vspace{8pt}
    \small 
    \faIcon{github}
    \href{https://github.com/arpitdwi}{\underline{github.com/arpitdwi}} $  $
    \faIcon{linkedin}
    \href{https://linkedin.com/in/arpitdwi}{\underline{linkedin.com/in/arpitdwi}} $  $
    \faIcon{envelope}
    \href{mailto:arpitdwived@gmail.com}
    {\underline{arpitdwived@gmail.com}}
    \faIcon{phone-alt}
    {\underline{+91-8976058580}}
\end{center}

% -------------------- EDUCATION --------------------
\vspace{-5pt}
\section{Education}
  \resumeSubHeadingListStart
  
    \resumeSubheading
      {Indian Institute of Technology Kharagpur}{2017 - 2022}
      {B.Tech + M.Tech (Dual Degree)}{CGPA: 9.2/10.0}
      \resumeItemListStart
      \resumeItem{\textbf{Major:} Electronics and Electrical Communication Engineering}
      \resumeItem{\textbf{Minor:} Computer Science and Engineering}
      \resumeItem{\textbf{Micro:} Entrepreneurship and Innovation}
      \resumeItemListEnd
      
      
\vspace{5pt}      

    \resumeSubheading
      {Delhi Public School, Navi Mumbai}{2017}
      {All India Senior School Certificate Examination}{Marks: 94.6\%}
      
\vspace{5pt}

    \resumeSubheading
      {Delhi Public School, Navi Mumbai}{2015}
      {All India Secondary School Examination}{CGPA: 10/10}

  \resumeSubHeadingListEnd

% -------------------- EXPERIENCE --------------------
\vspace{-5pt}
\section{Experience}
  \resumeSubHeadingListStart

      \resumeProjectHeading
        {\textbf{Data and Applied Scientist, Bing Ads, Microsoft}}{Jun 2022 - Present}
        \resumeItemListStart
            \resumeItem{Fine-tuned XLM-Roberta based multilingual models pretrained on ad specific data for predicting the relevance between query-ad pairs with ROC-AUC score of 91.96\%, marking a 4-point improvement from the baseline model}
            \resumeItem{Innovated GPT prompts to generate relevance labels resulting in improvement of over 5 AUC points compared to human labels and enabling the transition from human-labeled training data to GPT-generated labels}
            \resumeItem{Explored architectures for long document understanding such as Reformer and Longformer to facilitate comprehensive understanding of landing pages and significantly reduce training time}
            \resumeItem{Carried out knowledge distillation to train compact BERT models and DNNs for efficient online deployment}
            \resumeItem{Engineered novel features and trained models using curriculum learning strategies to enhance model robustness}
        \resumeItemListEnd

      \resumeProjectHeading
        {\textbf{Data and Applied Scientist Intern, Bing Ads, Microsoft}}{May 2021 - Jul 2021}
        \resumeItemListStart
            \resumeItem{Worked on identifying spoof domains in ads using Siamese Convolutional Neural Networks}
            \resumeItem{Curated a comprehensive dataset for training using manual generation techniques and achieved a validation ROC-AUC score of 98.93\% using the Triplet loss function}
            \resumeItem{Leveraged Faiss for efficient nearest-neighbor searches, enabling the rapid identification of existing spoof URLs which could subsequently be used as a dataset}
        \resumeItemListEnd

      \resumeProjectHeading
        {\textbf{Computer Vision Intern, Proof of Performance Data Services Pvt Ltd}}{Apr 2020 - Jun 2020}
        \resumeItemListStart
            \resumeItem{Deployed You Only Look Once  algorithm to develop a tree detection system, enabling the identification of all trees within a given image or video and classified them with an accuracy of 88\% using convolutional neural networks}
            \resumeItem{Implemented non-maximum suppression and performed image processing using OpenCV to filter out the false positives post tree detection leading to an increase in precision}
        \resumeItemListEnd

      \resumeProjectHeading
        {\textbf{Natural Language Processing Intern, Glasssquid.io}}{Oct 2019 - Jan 2020}
        \resumeItemListStart
            \resumeItem{Built a classifier to distinguish between IT and non-IT jobs using their job descriptions with an accuracy of 93\% using TF-IDF weighted Word2Vec and XGBoost}
            \resumeItem{Utilized SQL for data extraction and analysed the trends in skills and experience of users, their preferred job roles, location etc. to get valuable business insights}
        \resumeItemListEnd
          
    \resumeSubHeadingListEnd 

% -------------------- PAPERS AND PUBLICATIONS --------------------
\vspace{-5pt}
\section{Papers and Publications}

\begingroup
\renewcommand{\section}[2]{}%
%\renewcommand{\chapter}[2]{}% for other classes
\begin{thebibliography}{}
\bibitem{p1} Rima Hazra, Arpit Dwivedi and Animesh Mukherjee. Is this bug severe? A text-cum-graph based model for bug severity prediction. \emph{Joint European Conference on Machine Learning and Knowledge Discovery in Databases}, pp. 236-252. Springer, 2022. 

\bibitem{p2} Bishal Santra, Ravi Ghadia, Arpit Dwivedi, Manish Gupta and Pawan Goyal. CORAL: Contextual Response Retrievability Loss Function for Training Dialog Generation Models. \emph{Preprint,} \href{https://arxiv.org/abs/2205.10558v2}{arXiv:2205.10558v2}. 
\end{thebibliography}
\endgroup

% -------------------- SKILLS --------------------
\section{Skills}
 \begin{itemize}[leftmargin=0.15in, label={}]
    \small{\item{
    
     \textbf{Relevant Coursework}{: Algorithms, Machine Learning, Deep Learning, Matrix Algebra, Probability and Stochastic Processes, Image Processing, Big Data Processing, Natural Language Processing }
     
     \textbf{Languages and Libraries}{: C/C++, Python, SQL, \LaTeX, PyTorch, Hugging Face, Numpy, Pandas, Matplotlib, Scikit-Learn, OpenCV, NetworkX, PySpark, Streamlit}
     
     \textbf{Systems / Platforms}{: Git, Windows, Linux}
     
     % \textbf{Frameworks}{: React, Node.js, Flask, JUnit, WordPress, Material-UI, FastAPI} \\
     
     % \textbf{Libraries}{: pandas, NumPy, Matplotlib}
     
    }}
 \end{itemize}

% -------------------- PROJECTS --------------------
\vspace{-5pt}
\section{Projects and Competitions}
    \resumeSubHeadingListStart

        \resumeProjectHeading
        {\textbf{Discourse mutual information based evaluation metric}}{Mar 2022 - May 2022}
        \resumeItemListStart
            \resumeItem{Proposed a novel unreferenced metric for dialogue evaluation using transformer encoders pre-trained using Discourse Mutual Information based loss function that enabled the model to capture intricate relationships}
            \resumeItem{Created a response pool for scoring and outperformed strong baselines on correlation with human judgements}
          \resumeItemListEnd
    
        \resumeProjectHeading
        {\textbf{Multilingual news article similarity }}{Mar 2022 - Apr 2022}
        \resumeItemListStart
            \resumeItem{Measured the similarity between mono/cross lingual news article pairs by fine tuning encoders and stacking}
            \resumeItem{Used mBERT models for predicting six different labels such as geography, narrative, style etc. and an artificial neural network (ANN) as the meta model to get the final similarity score}
            \resumeItem{Utilized Mean Squared Error (MSE) loss during model training, resulting in a notable Pearson’s Correlation Coefficient of 0.702 on the final evaluation dataset }
          \resumeItemListEnd
          
      \resumeProjectHeading
        {\textbf{RL based training of generative dialogue systems }}{Jul 2021 - Feb 2022}
        \resumeItemListStart
            \resumeItem{Trained transformer based generative dialogue systems using scores from retrieval dialogue systems instead of using traditional cross entropy loss based methods that solely rely on the ground truth response}
            \resumeItem{Used scores from retrieval models like BERT and ESIM as reward for training the model using Reinforce algorithm}
            \resumeItem{The final model demonstrated significant improvements in response quality upon evaluation in terms of diversity as well as coherence with the context as compared to the conventional generative dialogue systems}
        \resumeItemListEnd

      \resumeProjectHeading
        {\textbf{AbInBev Maverick 2.0 Hackathon }}{Apr 2021 - May 2021}
        \resumeItemListStart
            \resumeItem{Developed an application to recommend customized on-invoice and off-invoice discounts. Reached the grand finale among 750+ competing teams and was applauded by the panelists for outstanding approach}
            \resumeItem{Employed feature engineering and outlier detection techniques before training a combination of classification and regression models to accurately predict each component of the discount using an ensemble of gradient boosted trees}
        \resumeItemListEnd

      \resumeProjectHeading
        {\textbf{Early identification of severe bugs in Ubuntu}}{Jan 2021 - Apr 2021}
        \resumeItemListStart
            \resumeItem{Predicted the severity of new bugs in advance using text, graph and metadata-based features with 74\% accuracy }
            \resumeItem{Constructed various networks such as a bug-bug network based on affected packages using the NetworkX library and extracted graph features like Degree, PageRank, and Clustering Coefficient}
            \resumeItem{Used Doc2Vec and SBERT for learning the text features present in the bug descriptions and comments}
        \resumeItemListEnd

      \resumeProjectHeading
        {\textbf{Imposter Detection}}{Oct 2020 - Nov 2020}
        \resumeItemListStart
            \resumeItem{Authenticated users on the basis of their mouse activity using features like click time, pause time, cursor velocity}
            \resumeItem{Used a supervised self organizing map (SOM) on top of an unsupervised SOM to make predictions with 83\% recall}
        \resumeItemListEnd
        
    \resumeSubHeadingListEnd


% -------------------- AWARDS AND ACHIEVEMENTS --------------------
\vspace{-5pt}
\section{Awards and Achievements}
      \resumeItemListStart
      \resumeItem{Secured All India Rank 946 in JEE Advanced 2017 among 150K+ students who appeared for the examination}
      \resumeItem{Secured All India Rank 1002 in JEE Mains 2017 among nearly 1.2M students who appeared for the examination}
      \resumeItem{Certified for being in the top 1 percentile in National Standard Examination in Physics held in November 2016}
      \resumeItem{Secured All India Rank 192 in Kishore Vaigyanik Protsahan Yojana 2015 conducted by Indian Institute of Science}
      \resumeItem{Cleared National Talent Search Examination in Class X and was one of the 775 students selected for the scholarship}
      \resumeItemListEnd

\end{document}
